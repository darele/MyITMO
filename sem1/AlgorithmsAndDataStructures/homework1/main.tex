\documentclass{article}
\usepackage[T2A]{fontenc}
\usepackage[utf8]{inputenc}
\usepackage[russian]{babel}
\usepackage{amsmath}
\usepackage{amssymb}
\usepackage[makeroom]{cancel}
\usepackage{hyphenat}
\hyphenation{ма-те-ма-ти-ка вос-ста-нав-ли-вать}
\usepackage{graphicx}

\title{Домашнее Задание №1}
\author{Дарвин Эдлеазар Пиче Круз}

\begin{document}

\maketitle

\section{}

Для каждой из приведенных ниже программ найдите и аргументируйте точную $\mathcal{O}$ асимптотику времени ее работы

\begin{tabbing}
    \hspace{1 cm} \= \hspace{1 cm}  \= \hspace{1 cm} \= \hspace{1 cm} \= \hspace{1 cm}\\
\> (a) for i = 0..n:\\
\>\> for j = 0..i:\\
\> \> \> for k = 0..j:\\
\>\>\>\> print(i, j, k)
\end{tabbing}

\textbf{Решение}

Первый цикл выполняет $n$ операций, в каждом из них второй цикл делает $i$ операций значит:
\begin{equation*}
    1+2+3+...+n
\end{equation*}
И третий цикл делает $j$ операции для каждой итерации второго цикла, значит:
\begin{equation*}
    (1(1)+2(2)+3(3)+...+n(n)) = (1^2+2^2+...+n^2) = \dfrac{n(n+1)(2n + 1)}{6}
\end{equation*}
очевидно что он работает в $\mathcal{O}(n^3)$

\begin{tabbing}
    \hspace{1 cm} \= \hspace{1 cm}  \= \hspace{1 cm} \= \hspace{1 cm} \= \hspace{1 cm}\\
\> (b)for i = 0..n:\\
\>\>j = 0, k = 2 $\cdot$ i\\
\>\>while j < k:\\
\>\>\>j++, k- -\\
\end{tabbing}

\textbf{Решение}

Первый цикл выполняет $n$ операций, в каждом из них второй цикл делает $i$ операций (так как 
$k = 2\cdot i$ но j и k привлижаются друг к другу 2 два блока в одной итерации) значит:
\begin{equation*}
    1+2+3+...+n = \dfrac{n(n + 1)}{2}
\end{equation*}

очевидно что оно работает в $\mathcal{O}(n^2)$

\begin{tabbing}
    \hspace{0.5 cm} \= \hspace{0.5 cm}  \= \hspace{0.5 cm} \= \hspace{0.5 cm} \= \hspace{0.5 cm}\\
\> (c)i = 1\\
\>\>while i < n:\\
\>\>\>for j = 0..i:\\
\>\>\>\>print(i, j)\\
\>\>\>i = i $\cdot$  2
\end{tabbing}

\textbf{Решение}

Первый цикл выполняет $\log n$ операций (так как каждый раз $i$ увеличивает два раза, значит приближается к $n$ два раза быстрее),
 в каждом из них второй цикл делает $i$ операций значит:
\begin{equation*}
    1+2+4+...+2^{\log n} = 2^{\log (n) + 1} - 1 = 2\cdot 2^{\log n} - 1 = 2n - 1
\end{equation*}

очевидно что оно работает в $\mathcal{O}(n)$

\begin{tabbing}
    \hspace{0.5 cm} \= \hspace{0.5 cm}  \= \hspace{0.5 cm} \= \hspace{0.5 cm} \= \hspace{0.5 cm}\\
\> (d)for i= 0..n \\
\>\>j = i\\
\>\>\>while j > 0:\\
\>\>\>\>j = j / 2\\

\end{tabbing}

\textbf{Решение}

Первый цикл выполняет $n$ операций, 
 в каждом из них второй цикл делает $\log i$ операций значит:
\begin{equation*}
    \log (1) + \log (2) + \log (3) \cdots + \log (n)
\end{equation*}

изпользуя метод интегралами:

\begin{equation*}
    \log (1) + \log (2) + \log (3) \cdots + \log (n) = \int_{1}^{n} log(x) dx
\end{equation*}
\begin{equation*}
    \int_{1}^{n} log(x) dx = \dfrac{n \cdot \log (n) - 1}{log(2)} = \dfrac{n\cdot \log (n) - n}{\log (2)}
\end{equation*}
Алгоритм работает в $\mathcal{O}(n \log n)$

\bigskip
\bigskip
\section{}
Докажите следующие соотношения по определению (выберите константы $c$ и $n_0$ и докажите соответствующее неравенство)
\begin{equation*}
    (a) \;\; \log (n) = \Omega(20)
\end{equation*}

\textbf{Решение}

Возьмём $n_0 = 2$ и $c = \dfrac{1}{20}$, тогда:

\begin{equation*}
    \log (n) \geq \dfrac{1}{20}\cdot 20 \; \; \forall n_0 > 1
\end{equation*}

\newpage
\begin{equation*}
    (b) \;\; 2^{n} = \mathcal{O}(3^n)
\end{equation*}

\textbf{Решение}

Возьмём $n_0 = 1$ и $c = 1$, тогда:

\begin{equation*}
    2^n \leq 3^n \; \; \forall n_0 \geq 1
\end{equation*}

\bigskip
\bigskip
\begin{equation*}
    (c) \;\; n(n - 8) = \Omega(n^2)
\end{equation*}

\textbf{Решение}

Возьмём $n_0 = 9$ и $c = 1/81$, тогда:

\begin{equation*}
    n^2 - 8n \geq \dfrac{n^2}{81} \;\; \forall n_0 \geq 9
\end{equation*}


\bigskip
\bigskip
\begin{equation*}
    (d) \;\; 3n + 2\sqrt{n} = \mathcal{O}(n \log n)
\end{equation*}

\textbf{Решение}

Возьмём $n_0 = 2$ и $c = 10$, тогда:

\begin{equation*}
    3n + 2\sqrt{n} \leq 10n \log n \;\; \forall n > 1
\end{equation*}




\bigskip
\bigskip
\begin{equation*}
    (e) \;\; n! = \Omega (5^n)
\end{equation*}

\textbf{Решение}

Возьмём $n_0 = 1$ и $c = \dfrac{1}{48828125}$, тогда:

\begin{equation*}
    n! \geq \dfrac{5^n}{48828125} \;\; \forall n >= 1
\end{equation*}

\section{}
Докажите, что если $f(n) = \mathcal{O}(h(n))$ и $g(n) = \mathcal{O}(h(n))$, то $f(n) + g(n) = \mathcal{O}(h(n))$

\textbf{Решение}

так как $f(n) = \mathcal{O}(h(n)) \rightarrow f(n) <= c_1\cdot h(n)$ и одновремено $g(n) = \mathcal{O}(h(n)) \rightarrow g(n) <= c_2\cdot h(n)$
можно считать что:

\begin{equation*}
    f(n) + g(n) <= (c_1 + c_2)h(n)
\end{equation*}

Возьмём $c_3 = c_1 + c_2$ тогда:

\begin{equation*}
    f(n) + g(n) <= c_3\cdot h(n) \rightarrow f(n) + g(n) = \mathcal{O}(h(n))
\end{equation*}

\section{}

Докажите по индукции, что если $T(n) = 2T\left(\dfrac{n}{4} \right) + \log_{2}(n) \longrightarrow T(n) = O(\log^2 n)$

\textbf{Решение}

База индукции

\begin{equation*}
    T(1) = 1\\
\end{equation*}
\begin{equation*}
    T(4) = 2(1) + \log_{2}(4) = 4 <= c\log^2(4)
\end{equation*}

\begin{equation*}
    T(16) = 2(4) + \log_{2}(16) = 12 <= c\log^2(8)
\end{equation*}
при $c \geq 3$\\
Пусть для некоторых $n$
\begin{equation*}
    T(n / 4) + \log_{2}n <= c\cdot log^2(4)
\end{equation*}
переход:
\begin{equation*}
    T(n + 1)=2T\left(\dfrac{n+1}{4}\right) + \log_{2}(n + 1)
\end{equation*}
\begin{equation*}
    = 2\left(2T\left(\dfrac{n+1}{8}\right) + \log_{2}\left(\dfrac{n+1}{4}\right) \right) + \log_{2}(n + 1)
\end{equation*}

очевидно что $2\left(2T\left(\dfrac{n+1}{8}\right) + \log_{2}\left(\dfrac{n+1}{4}\right) \right) \leq c_2 \cdot \log^2 (n + 1)$ по скольку меньше чем $c_1 \cdot \log^2 \left(\dfrac{n + 1}{4}\right)$, тоже очевидно, что $\log_{2} n <= c2 \log^2 (n + 1)$ при $c_2$ достаточно большой. Тогда:
\begin{equation*}
    T(n + 1) = \mathcal{O}(\log^2 (n + 1))
\end{equation*}
\end{document}
