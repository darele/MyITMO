\documentclass{article}
\usepackage[T2A]{fontenc}
\usepackage[utf8]{inputenc}
\usepackage[russian]{babel}
\usepackage{amsmath}
\usepackage{amssymb}
\usepackage[makeroom]{cancel}
\usepackage{hyphenat}
\hyphenation{ма-те-ма-ти-ка вос-ста-нав-ли-вать}
\usepackage{graphicx}
\usepackage{listings}

\title{Практика №5}
\author{Дарвин Эдлеазар Пиче Круз}

\begin{document}

\maketitle

\section*{5.8}

В игре есть $n$ типов ресурсов, для постройки одного юнита требуется $a_i$ единиц ресурса $i$ для всех $i$. У игрока есть bi единиц ресурса $i$ для каждого $i$, и еще $c$ единиц золота.
Одну единицу золота можно обменять на $d_i$ единиц ресурса i для произвольного $i$.
Сколько юнитов может построить игрок?

\section*{Решение}

\begin{lstlisting}
    arr = max_heap of pairs
    ans = 0
    for i in 0..n:
        //What I can get with my current resources
        ans += (b[i] / a[i])
        
        //how close I am to get another Uton for 
        //every i if I add d[i] (use a gold piece)
        //(positive results are even better :))
        arr.push([d[i] - (b[i] - (a[i] % b[i])), i]);
    
    while(c > 0):
        //always trade the cheapest resource
        t = arr.top().first
        i = arr.top().second
        arr.pop();
        if (t >= 0):
            ans += t / b[i]
            t -= ((t / b[i]) * b[i])
        
        //again, if I add d[i], how close 
        //am I from the next Uton?
        arr.push([d[i] - (b[i] - t), i]);
        
        c -= 1
        
    print(ans)
\end{lstlisting}

\end{document}
